\chapter{Технологический раздел}
\label{cha:impl}

% В данном разделе описано изготовление и требование всячины. Кстати,

\section{Выбор и обоснование языка программирования и среды разработки}

Язык программирования: C++, glsl

Библиотеки: vulkan, imgui (gui библиотека), glm (математическая библиотека), libxcb (linux)

Редактор: VSCode

Система автоматизации сборки: CMake

% C++ use for interacting with user, glsl get compile to 


\begin{itemize}
    \item Vukan: графический и вычислительный API нового поколения,
    обеспечивающий высокоэффективный кроссплатформенный доступ к
    современным графическим процессорам, используемым в самых разных устройствах.

    \item ImGui: Графический пользовательский интерфейс немедленного режима,
    представляет собой шаблон проектирования графического пользовательского интерфейса,
    который использует графическую библиотеку немедленного режима
    для создания графического интерфейса.

    \item libvkbase: Поскольку время ограничено, а Vulkan - это графический api
    низкого уровня, я буду использовать репозиторий на github, используя некоторые
    очень хорошо абстрактные файлы (рекомендованные создателями vulkan)
    (только те, которые мне нужны). Оттуда я создаю статическую библиотеку libvkbase и использую ее.
    
    \item Программное обеспечение было разработано для Linux, но его можно
    полностью расширить как для MacOS, так и для Windows.
\end{itemize}

\section{Описание структуры программы}

\begin{itemize}
    \item data/: Содержит скомпилированные шейдеры и шрифты, требуется при запуске программы.
    \item lib/: Содержит необходимые библиотеки.
    \item shaders/: Содержит исходный код вычислительных шейдеров.
    \begin{itemize}
        \item ray/
        \begin{itemize}
            \item constant.glsl
            \item entity.glsl : Описывает структуры данных, хранящиеся в графическом процессоре.
            \item intersect.glsl : Обработка пересечений
            \item rt.comp : Точка входа в вычисление шейдера
            \item ...
        \end{itemize}
        \item ui/: Шейдеры для пользовательского интерфейса
    \end{itemize}
    \item src/: Содержит исходный код C++
    \begin{itemize}
        \item Model/
        \begin{itemize}
            \item camera.h
            \item entity.h
            \item material.h
            \item scene.h, scene.cpp: Управлять всеми объектами в сцене.
        \end{itemize}
        \item vulkanbase.h, vulkanbase.cpp: Вещи, которые нужны базовому приложению vulkan
        \item app\_ui.cpp: Пользовательский интерфейс
        \item app\_vulkan.cpp: Обработка процессов, связанных с vulkan
        \item main.cpp
        \item ...
    \end{itemize}
    \item vkbase/: Файлы, используемые для сборки libvkbase. Отредактировано и усечено для уменьшения размера файла.
    \item CMakeLists.txt
    \item run.sh: Скрипт помогает быстро собрать, запустить и протестировать
\end{itemize}

Программный продукт состоит из файла \textbf{app} и папки \textbf{data/}, содержащей шрифты и шейдеры.

\section{Интерфейс программы}

Я использую Dear Imgui - библиотеку графического пользовательского интерфейса для C ++,
она выводит оптимизированные буферы вершин (в данном случае визуализируется в графическом интерфейсе приложения vulkan).

\begin{itemize}
    \item Отладочная информация
    \begin{itemize}
        \item Видеокарта
        \item Кадров в секунду (fps)
    \end{itemize}
    \item Менеджер сцены
    \begin{itemize}
        \item Свет: может менять положение, траекторию движения
        \item Камера: может менять положение
        \item Список объектов в сцене
    \end{itemize}
    \item Менеджер объектов
    \begin{itemize}
        \item Добавить объект: сфера, плоскость, куб
        \item Изменить объект: в зависимости от выбранного объекта могут быть
    изменены соответствующие параметры
        \item Изменить материал: $k_a, k_s, k_d, \alpha, ior$
    \end{itemize}
\end{itemize}
