\chapter*{Заключение}
\addcontentsline{toc}{chapter}{Заключение}

В ходе реализации анализируются алгоритмы синтеза 3d-изображений
и выбирается наиболее подходящий алгоритм - трассировка лучей.
В исследовании рассматриваются основы создания простого трассировщика лучей.
Используя сцен коробки Корнелла, программа хорошо представляет физические свойства света, и свойства материалов.
Программные продукты разработаны таким образом, чтобы пользователи
могли создавать, изменять положение объектов в комнате, изменять свойства материалов,
изменять траекторию источника света, изменять положение камеры.
Программа моделирует физическое явление отражения, преломления и приближенно моделирует рассеяние.
\\

Во время выполнения задачи узнал больше о том, как работать с низкоуровневым графическим API
(vulkan), а также как писать программы, работающие на графическом процессоре (шейдер).
Узнал немного, как работает графический конвейер, получил больше знаний о компьютерной графике,
пересмотрел объектно-ориентированное программирование и C ++.
