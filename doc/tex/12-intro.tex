\chapter*{Введение}
\addcontentsline{toc}{chapter}{Введение}


Компьютерная графика - основная технология в цифровой фотографии, кино, видеоиграх,
дисплеи телефона и компьютера,
а также многие другие приложения.
Разработано специализированное оборудование и программное обеспечение,
с дисплеями большинства устройств, управляемых оборудованием компьютерной графики.
В наши дни технологии быстро развиваются, компьютеры становятся все более мощными.
Из-за этого мы ожидаем чего-то более реалистичного в видеоигре или фильме, картинке.
Следовательно, многие методы рендеринга, основанные на физике, развиваются и продолжают развиваться.
\\

Целью данной работы является реализация простого физически основанного рендерера,
рендеринга на графическом процессоре.
Рендерер должен демонстрировать следующие оптические свойства:
отражение, преломление.
Некоторые основные материалы: пластик, стекло, металл.
\\

% Computer graphics is a core technology in digital photography, film, video games,
% phone and computer displays, and many other applications.
% Specialized hardware and software has been developed,
% with the displays of most devices being driven by computer graphics hardware.
% Nowaday, technology evolve quickly, computer become more and more powerfull.
% Because of this, we're expecting something that looks more realistic in the video game or movie, picture.
% Consequently, many physics-based rendering techniques are and continue to evolve.

% Purpose of this work is implementing a simple physically based renderer, render on gpu.
% Renderer must show these optical properties: reflect, refract, specular, (fernel effect)... .
% Some basic materials: plastics, glass, metal, (rough rock, take long time to render).
% Since i didn't know yet how long it's take to render 1 frame of photorealistic image,
% i will try in two ways, then pick better:

% \begin{itemize}
%     \item Simple renderer, it's must fast, with dynamic scene (minimum 30fps on gpu gtx1050).
%     \item Offscreen renderer with rough materials and global illumination which take long time to render.
% \end{itemize}

% (Progressing renderering is the best choice but hard for me to implement.)\\


% So, the subtask are:
Итак, подзадачи:

\begin{itemize}
    \setlength{\itemsep}{0em}
    \item Анализ объектов на сцене
    \item Анализ алгоритмов удаления невидимыхповерхностей
    \item Выбор модели освещения
    \item Алгоритм построения теней
    \item Произвести основные математические расчеты для реализации выбранных алгоритмов
    \item Выбор языка программирования и среды разработки
    \item Реализовать интерфейс программного модуля
    \item Реализовать основной модуль программы (редактор сцены)
    \item Реализовать шейдеры (обработка данных для отображения на экране, работа на gpu)
    % \item Implement math, physics properties of materials
\end{itemize}
