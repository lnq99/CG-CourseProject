\chapter{Конструкторский раздел}
\label{cha:design}

% В данном разделе проектируется новая всячина.

\section{Общий алгоритм решения задачи}

\begin{enumerate}
\item Задать положение наблюдателя.

\item Задать объекты сцены.

\item Задать источники света и орбиту движения.

\item Для каждого кадра:

4.1 Обработка пользовательского ввода.

4.2 Обновить положение источника света, положение объекты.

4.3 Для каждого пикселя вызвать под процедуру raytrace, вычислить цвет этого пикселя, сохранить в изображение.

4.4 Отобразить изображение.
\end{enumerate}


\section{Алгоритм трассировки лучей}

\subsection*{Псевдокод}

Поскольку рекурсивный алгоритм на gpu не поддерживает (glsl), я немного изменил алгоритм следующим образом

% $
% for each pixel on the viewing area
%     for each primitive in the world model
%         if ray-object intersection
%             select the frontmost intersection
%     calculate color
%     trace the relection and refraction rays
%     blend two color components
% $

\begin{verbatim}
для каждого пикселя в области просмотра
    для каждого примитива
        если пересечение луча и объекта
            выбрать самый передний пересечение
    рассчитать цвет
    проследить лучи отражения и преломления
    смешать два цветовых компонента
\end{verbatim}


% \subsection*{Пересечение трассирующего луча с треугольником}
\pagebreak
\subsection*{Пересечение луча с сферой}

Параметрическая форма луча:
$P(t) = O + tD$

% Параметрические уравнения сферы:
Если точка P находится на сфере C радиуса r, то:
$(P-C)^2=r^2$

Если луч и сфера пересекаются, это уравнение должно иметь хотя бы одно решение:

$
(O + tD - C)^2 = r^2
$

$
t^2 \, D^2 + 2t \, D(O-C) + (O-C)^2 - r^2 = 0
$

После вычисления t получаем точку пересечения, мы можем вычислить нормаль к сфере в точке пересечения
$n = normalize(P-C)$


\subsection*{Пересечение луча с плоскостью}

$P_0$ - точка на плоскости, $N$ - вектор нормали к плоскости. Если точка $P$ лежит на плоскости, $P$ не совпадает с $P_0$, то:

$(P - P_0).N = 0$

Если луч пересекает плоскость в точке $P$:

$(O + tD - P_0).N = 0$

$t.D.N = (O-P_0).N$


\subsection*{Пересечение луча с треугольником}

Параметрическая форма луча: $P(t) = O + tD$

Параметрические уравнения плоскости: $Ax + By + Cz + E = 0$;\\
$N$ - вектор нормали

Решение точки пересечения луча и плоскости:

$t.D.N = -(N.O + E)$

Нужно еще выяснить, находится ли точка пересечения внутри треугольника.
Вычислить векторное произведение, определенного вершинами двух
ребер, и вектора, определенного вершиной и точкой первого ребра.
Вычислить скалярное произведение результирующего вектора и нормали многоугольника.
Знак полученного скалярного произведения определяет, находится ли точка
справа или слева от этого ребра. Повторять по каждому ребрам треугольника.
Нет необходимости проверять другие ребра, если один из них не прошел проверку.


\subsection*{Пересечение луча с кубом (aabb box)}

Выровненный по оси ограничивающий прямоугольник AABB может быть определен его минимальными и максимальными точками A и B.

AABB определяет набор плоскостей, перпендикулярных оси координат. Каждый компонент плоскости может быть определен следующим уравнением:

$
x = A_x, y = A_y, z = A_z
$

Точка, где луч пересекает одну из этих плоскостей:

$
O_x + t_x D_x = A_x
$

В векторной форме:

$t_A = (A - O) / D$ \quad $t_B = (B - O) / D$

Чтобы найти решение, которое действительно является пересечением с коробкой, требуется большее значение параметра t для пересечения в минимальной плоскости, требуется меньшее значение параметра t для пересечения в максимальной плоскости.


\subsection*{Отраженный луч}

Законы отражения: $\theta_i = \theta_r$

Отраженный луч $R$ рассчитывается следующим образом:\\

$
R = I - 2 (N.I)N
$

\subsection*{Преломленный луч}

Закон Снеллиуса: $n_1 sin(\theta_1) = n_2 sin(\theta_2)$

Преломленный луч $T$ рассчитывается следующим образом:\\

$n = \dfrac{n_1}{n_2}$\\

$c_1 = N.I$\qquad
$c_2 = \sqrt{1-n^2 sin^2(\theta)}$\\

$T = nI + (nc_1 - c_2)N$


% \subsection*{Распределение света световой точки}


% \begin{figure}[ht]
%     \centering
%     \includegraphics[width=0.8\textwidth]{img/light_sphere.svg}
%     \caption{Алгоритм трассировки лучей}
%   \end{figure}


\section{Модель отражения Блинна – Фонга}

Применять с одним источником света:\\

$
\displaystyle I_p = k_ai_a + k_d(L.N)i_d + k_s(N.H)^\alpha i_s
$\\

$
H = \dfrac{L + V}{||L + V||} \qquad (R.V = N.H)
$
\\\\
где:\\
$k_a, k_s, k_d$ - коэффициент фонового, зеркального, диффузного освещения\\
$i_s, i_d$ - интенсивности зеркальной и диффузной составляющих источника света\\
$i_a$ - интенсивности окружающего света от всех источников света\\
$\alpha$ - которая является константой яркости материала\\
$L$ - вектор направления от точки на поверхности к источнику света\\
$N$ - нормаль в этой точке на поверхности\\
$R$ - направление, в котором идеально отраженный луч света будет направлен из этой точки на поверхности\\
$V$ - направление, указывающее на зрителя.


\section{Выбор типов и структур данных}


\begin{itemize}
    \item Ray: origin (vec3), direction (vec3)
    \item Sphere: center (vec3), radius (float), material
    \item Plane: normal (vec3), distance (float) (расстояние до центра сцены)
    \item Cube (aabb box): два точки A (vec3) и B (vec3)
    \item Diamond: array triangle (vec3*3), material
    \item Material: ambient (vec3), diffuse (vec3), specular (vec3), shininess (int),
    reflectivity (float), transparent (bool), ior (float) (изменится в процессе реализации)
    % \item Material: (изменится в процессе реализации)
    % \begin{itemize}
    %     \item ambient (vec3)
    %     \item diffuse (vec3)
    %     \item specular (vec3)
    %     \item shininess (int)
    %     \item reflectivity (float)
    %     \item transparent (bool)
    %     \item ior (float)
    % \end{itemize}
\end{itemize}


% Источник света – задается расположением и направленностью света.

% Объекты сцены – задаются вершинами и гранями.

% Система частиц – хранит в себе частицы, направление движения

% Математические абстракции

% Точка – хранит координаты x, y, z

% Вектор – хранит направление по x, y, z

% Многоугольник – хранит вершины, нормаль, цвет

% Интерфейс – используются библиотечные классы для предоставления
% доступа к интерфейсу.
