\chapter*{Индивидуальное задание}


1. Техническое задание

Разработать программу моделирования движения объектов из заданного набора методом ключевых кадров. Количество ключевых кадров задается пользователем и не может превышать 300. Для каждого ключевого кадра задается положение движущегося объекта или группы объектов. Моделирование движения должно осуществляться с использованием операций переноса, масштабирования, поворота. Для каждого промежуточного кадра программа должна рассчитывать текущее положение каждого объекта. Должен быть задан набор законов управления движением при переходе между двумя соседними ключевыми кадрами. Исследовать возможность учета освещенности и построения теней при движении объектов

2. Оформление курсового проекта

2.1. Расчетно-пояснительная записка на 25-30  листах формата А4.
Расчетно-пояснительная записка должна содержать постановку введение, аналитическую часть, конструкторскую часть, технологическую часть, экспериментально-исследовательский раздел, заключение, список литературы, приложения.

2.2. Перечень графического материала (плакаты, схемы, чертежи и т.п.). На защиту проекта должна быть представлена презентация, состоящая из 15-20 слайдов. На слайдах должны быть отражены: постановка задачи, использованные методы и алгоритмы, расчетные соотношения, структура комплекса программ, диаграмма классов, интерфейс, характеристики разработанного ПО, результаты проведенных исследований.



3. Примечание:
    1. Задание оформляется в двух экземплярах; один выдаётся студенту, второй хранится на кафедре.

Дополнительные указания по проектированию
Моделируемые объекты выбираются из следующего набора стандартных тел:  параллелепипед, призма трехгранная, сфера, пирамида (четырехгранная), конус, цилиндр, тор. Для представления сферы, цилиндра, конуса, тора использовать полигональную аппроксимацию. 
     Пользователь должен иметь возможность задавать количество аппроксимирующих граней.
Задание положения, размеров и ориентации объектов должно производиться с помощью мыши, а также точно с помощью ввода значений из текстовых окон.
    Пользователь должен иметь возможность добавить в сцену любые объекты из заданного набора, общее количество объектов не может быть менее пяти.
   Пользователь должен иметь возможность выбирать закон управления движением для каждого интервала из определенного перечня (равномерный, скачкообразный, параболический).
   Провести исследование возможности добавления в сцену источников освещения и построения теней, отбрасываемых движущимися объектами, без потери ощущения непрерывного движения. Каждый источник освещения задается своим положением, цветом, интенсивностью.



Intro




% Во введении обосновывается актуальность выбранной темы (со ссылками на монографии, научные статьи), формулируется цель проекта («Целью проекта является…») и перечисляются задачи, которые необходимо решить для достижения этой цели («Для достижения поставленной цели необходимо решить следующие задачи…»).
% Среди задач, как правило, выделяют аналитические, конструкторские, технологические и экспериментальные. Решение этих задач описывается в соответствующих разделах.

% Рекомендуемый объем введения 1 - 2 страницы.


\chapter{Аналитический раздел}
\label{cha:analysis}
%
% % В начале раздела  можно напомнить его цель
%
В данном разделе анализируется и классифицируется существующая всячина и пути создания новой всячины. А вот отступ справа в 1 см. "--- это хоть и по ГОСТ, но ведь диагноз же...

\section{Анализ того и сего}

% Обратите внимание, что включается не ../dia/..., а inc/dia/...
% В Makefile есть соответствующее правило для inc/dia/*.pdf, которое
% берет исходные файлы из ../dia в этом случае.

\begin{figure}
  \centering
  % \includegraphics[width=\textwidth]{inc/dia/rpz-idef0}
  \caption{Рисунок}
  \label{fig:fig01}
\end{figure}

\begin{figure}
  \centering
  % \includegraphics[height=0.85\textheight]{inc/img/leonardo}
  \caption{Предполагаемый автопортрет Леонардо да Винчи}
  \label{fig:leonardo}
\end{figure}

В \cite{Pup09} указано, что...

Кстати, про картинки. Во-первых, для фигур следует использовать \texttt{[ht]}. Если и после этого картинки вставляются <<не по ГОСТ>>, т.е. слишком далеко от места ссылки, "--- значит у вас в РПЗ \textbf{слишком мало текста}! Хотя и ужасный параметр \texttt{!ht} у окружения \texttt{figure} тоже никто не отменял, только при его использовании документ получается страшный, как в ворде, поэтому просьба так не делать по возможности.

\section{Существующие подходы к созданию всячины}

Известны следующие подходы...

\begin{enumerate}
\item Перечисление с номерами.
\begin{enumerate}
\item Номера второго уровня.
\end{enumerate}
\item По мнению Лукьяненко, человеческий мозг старается подвести любую проблему к выбору
  из трех вариантов.
\end{enumerate}

Теперь мы покажем, как изменить нумерацию на «нормальную», если вам этого захочется. Пара команд в начале документа поможет нам.

\renewcommand{\labelenumi}{\arabic{enumi})}
\renewcommand{\labelenumii}{\asbuk{enumii})}

\begin{enumerate}
\item Изменим нумерацию на более привычную...
\item ... нарушим этим гост.
\begin{enumerate}
\item Но, пожалуй, так лучше.
\end{enumerate}
\end{enumerate}




% табл.~\ref{tab:tabular} и~\ref{tab:longtable}.
% \code{tabular}

% \begin{table}[ht]
%   \caption{Пример короткой таблицы с коротким названием}
%   \begin{tabular}{|r|c|c|c|l|}
%   \hline
%   Тело      & $F$ & $V$  & $E$ & $F+V-E-2$ \\
%   \hline
%   Тетраэдр  & 4   & 4    & 6   & 0         \\
%   Куб       & 6   & 8    & 12  & 0         \\
%   \hline
%   Эйлер     & 666 & 9000 & 42  & $+\infty$ \\
%   \hline
%   \end{tabular}
%   \label{tab:tabular}
% \end{table}

% \begin{figure}
  %   \centering
  %   % \includegraphics[height=0.85\textheight]{inc/img/leonardo}
  %   \caption{Предполагаемый автопортрет Леонардо да Винчи}
  %   \label{fig:leonardo}
  % \end{figure}
  
  % В \cite{Pup09} указано, что...
  
  % Во-первых, для фигур следует использовать \texttt{[ht]}.
  % ужасный параметр \texttt{!ht} у окружения \texttt{figure}
  
% \texttt{110mm},\textbf{всей}


% А формула~\eqref{eq:fourierrow} имеет некоторый смысл.

% \begin{equation}
% \label{eq:fourierrow}
% f(x) = \frac{a_0}{2} + \sum\limits_{k=1}^{+\infty} A_k\cos\left(k\frac{2\pi}{\tau}x+\theta_k\right)
% \end{equation}

% $ k\frac{2\pi}{\tau} = k\omega$ "--- круговая частота гармонического колебания,\\


% \begin{equation}
% a= \begin{cases}
%  3x + 5y + z, \mbox{если хорошо} \\
%  7x - 2y + 4z, \mbox{если плохо}\\
% \end{cases}
% \label{F:F2}
% \end{equation}



% \begin{lstlisting}[caption={Алгоритм оценки дипломных работ}]
% def EvaluateDiplomas():
%     for each student in Masters:
%         student.Mark := 5
% \end{lstlisting}

% \begin{verbatim}
% a_b = a + b; // русский комментарий
% if (a_b > 0)
%     a_b = 0;
% \end{verbatim}




%%% Local Variables:
%%% mode: latex
%%% TeX-master: "rpz"
%%% End:
